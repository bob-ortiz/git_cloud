\documentclass[11pt, twocolumn]{article}

\usepackage[utf8]{inputenc}
\usepackage{amsmath}
\usepackage{amssymb}
\usepackage{graphicx}
\usepackage{multicol}

% Margins
\topmargin=-0.45in
\evensidemargin=0in
\oddsidemargin=0in
\textwidth=6.5in
\textheight=9.0in
\headsep=0.25in

\title{DiD II}
\author{Gustavo A. Castillo Alvarez}
\date{Octubre 28, 2021}

\begin{document}
\maketitle	

% Optional TOC
% \tableofcontents
% \pagebreak

%--Paper--

\section{Section 1}
\subsection{Forma funcional}
El supuesto de \textbf{tendencias paralelas} es muy sensible a la forma funcional de la variable dependiente
Una transformacion puede cambiar todo: e.g. tener var. indep. como empleo, o log(empleo)
\begin{itemize}
    \item  El caso de \textit{empleo} estamos hablando en cambios absolutos
    \item El caso de \textit{log(empleo)} estamos hablando en cambios porcentuales
\end{itemize}
¿Por qué hay que tener mucho cuidado con las transformaciones en las variables en DiD?


\section{Diferencias en Diferencias en Diferencias (DDD)}
Si no se me cumple el supuesto de \textbf{tendencias paralelas} podría solucionar el problema con una 3ra diferencia.
e.g. Muralidharan and Prakash (2017): ¿Cómo diseñamos la identificación de este problema? La manera más sencilla sería utilizar el grupo de hombres que no recibieron la bicicleta como mi grupo de control, y así usando DiD puedo comparar cohortes de mujeres antes y después del tratamiento usando hombres como grupo de control.

\textbf{Problema:} Las tasas de cobertura de hombres y mujeres NO siguen la misma tendencia, probablemente NO hay tendencias paralelas: la tasa de cobertura de las mujeres a través del tiempo \textit{ha ido creciendo} a través del tiempo, NO se ha movido de manera paralela con la tasa de cobertura de los hombres.

\textbf{Solución:} La \textit{brecha} o las diferencias \textit{en la tendencia} sean iguales (parelelas) en otros estados.
Es posible que haya \textbf{tendencias paralelas} \textit{entre estados} en la \textbf{brecha} de cobertura en ausencia del tratamiento.
\[
DDD=DiD_{Bihar}-DiD_{Jharkhand}
\]


\section{Estudios de Evento}
Debemos utilizarlos cuando las unidades reciben el tratamiento den \textit{diferentes} momentos del tiempo. \textbf{Suponemos} que el evento corresponde a un \textbf{absorving state}.

Intuición: ¿Por qué no simplemente utiliar TWFE y permitir que la variable que determina el tratamiento $D_{it}$ se active en distintos momentos del tiempo?

Excepción es cuando tenemos \textbf{efectos constantes} y homogéneos (entre unidades y el tiempo), pues se cumple tendencias paralelas y podemos estimar la ecuación y el estimador $\tau$ por MCO sí captura el ATT.

Problema cuando tenemos presencia de: 
\begin{enumerate}
    \item Efectos heterogéneos
    \item Efectos dinámicos
\end{enumerate}

Si las unidades están recibiendo el tratmiento en distintos instantes del tiempo, temprano (k) y tarde (l) el problema radica en lo que $\tau$ está capturando, puesto que tenido estos "3 grupos" de untreated (u), early treated (k) and late trated (l) pueden crear 4 diferentes escenarios de DiD de 2x2 entre ellos:
Epa
\begin{multicols}{2}
\begin{itemize}
\item early vs untreated
\item late vs untreated
\item early vs late in $t_0$ 
\item late vs early in $t_1$
\end{itemize}
%\columnbreak
\end{multicols}

Los pesos son una función del \textbf{tamaño} de cada grupo y del \textbf{tiempo de exposición} al tratamiento. Así surgen dos problemas:
\begin{enumerate}
 \item Efectos heterogéneos: La distribución temporal del tratamiento afecta el resultado, es decir, los early están expuestos más al tratamiento que los late. 
 \item Efectos dinámicos: Algunos pesos pueden ser \textit{negativos}, por el 4to escenario de late vs early in $t_1$ puesto que se está usando un grupo \textit{tratado} en período post como un grupo de control. 

 \end{enumerate}

Esto es un problema de estimación, no un problema de la lógica de la identificación.
\subsection{Alternativas de Solución}
Utilizar 3 diferentes estimadores que buscan garantizar que los grupos de control que utilizaremos son los 
\begin{itemize}
    \item Callaway and Sant'Anna (2021): Solo escoger las grupos (e.g. los 2x2) que me interesan estimandolos de forma no-paramétrica, y luego incorporándolos "adecuadamente" (unos pesos bien comportados)
    \item Abraham and Sun (2021): 
    \item de Chaisemartin and D'Haultfoeuille (2021): No 
        
\end{itemize}

\subsection{Ejemplo Kleven et al. (2019)}
Busca estudiar el \textit{por qué} diferencias en aspectos laborales (ingresos, participación y horas trabajadas) entre hombres y mujeres y explicándolo con el efecto de tener hijos $T$ sobre la carrera de las mujeres.

Evento: momento (relativo) del nacimiento del primer hijo ($t=0$ es año en que nace el primer hijo)
Control: Hombres
Tratamiento: Mujeres

El efecto que se desa estimar es entonces
\[
    P_t=\frac{\tau^m-\tau^w}{E()}
.\] 

\end{document}

